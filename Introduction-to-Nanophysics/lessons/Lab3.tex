\documentclass[../main/main.tex]{subfiles}

\newdate{date}{2}{06}{2020}

\begin{document}

\section{Laboratory 3 }

\displaydate{date}. Compiled: \today. Alice


\subsubsection{Slide 48}
The last technique is \textbf{electron microscopy}, which is able to direct see the nanoparticle produced, so that we will be not constrained by any assumption or any model that we do for extracting information from a given characterization, but we will see directly the particle (we will measure directly its size and we will do the size distribution to obtain a statistical meaninful size distribution in our system).

\subsubsection{Slide 49}
Why electronmicroscopy? It is a technique able of atomic resolution. This is a trasmittion electron microscope image of a gold nanoparticle in silica. We can clearly see the atomic resolution. We see the single atoms in the cristalline particles with respect to the amorphous silica substrate around. So we can directly have a feeling of the degree of order of our nanostructures embedded in the matrix.

In the laboratory activity we will not deal with trasmittion electron microscope, which is the state of the art of the resolution, instead we will use \textbf{scanning electron microscope} which is another flavour of electronmicroscopy, in which the resolution is not tha powerful to obtain atomic resolution but which is sufficiently enough to obtain very accurate determination on the size, because we will have a resolution around  1nm, which is very useful to access the dimension of our nanoparticle.


\subsubsection{Slide 51}
We will be temped to use optical microscopy to look at our nanoparticle. Why do we not use photons? This is an optical microscope. We have ligth from below.
The resolution is limited, it is more or less of the order of \( lambda/2 \) for typical instruments.
The advantage is that this system is compact and work under ambient pressure.

The idea is to go beyond this limitation, which is related by the wavelength used. Why do not use photons with lower wavelength, that is X-rays? This could be a clever idea, but the limitation is that the refractive index of any material at those wavelengths is more or less 1, so there is no chance to reflect X-rays.

So the basic idea is: let us try to change the probe from photons to another object which exhibit an associated wavelength and which can be reflected and accelerated. We can use electrons.
For that reason electron microscopy is a very popular technique.

\subsubsection{Slide 52}
In general when you speak about resolution, you are temped to think about \textbf{magnification}. What is the resolution of your instrument so that you obtain a good magnification?

The lower the number of pixel is the information (it is the lower is the resolution).

In electron microscopi you want to resolve, that is you want to obtain information on the tinyest details in your sample, so that you can magnificy the image at a sufficient level to be seen by our eyes.
The resolving power of our eyes is around \( 0.1mm \).

\subsubsection{Slide 53}
Electrons can be easily produced. If we accelerated them trough a potential difference, they can reach a velocity that can be compared to the one of the speed of light quite easily.
So that relativistic expression for the energy should be considered (the energy with respect to the rest energy, the mass with respect the mass at rest...). What is important for us is the \textbf{DeBroglie wavelength} which is inversily proportional to the square root of the energy: the higher is the energy at which you accelerate your electrons, the lower is the wavelength.

Typical TEM accelerating voltages are in the range \( 200kV \).

The electromagneticts lenses are not so good as optical lenses. So for that reason, in the end we will see that the actual resolution that we can obtain with electrons with this energy will be around 1nm, which is quite good.

Electron microscope in trasmittion mode can have a resolution of angstrom.

\subsubsection{Slide 54}
This is a typical trasmission electron microscope.
It is a reversed path with respect to the optical microscope.

Inside the TEM there is vacuum (we do not want electron scattering).

In this slide: descrpition of TEM.

\subsubsection{Slide 55}
Another typical TEM. The height is due to the fact that you have a tower to accelerate electrons.
Modern micrscope prefer not to so high accelerating voltage, but better on improving the quality on the optics on the micrscope, so that you can obtain the same resolution (or even better) but without this elevated hight.


\subsubsection{Slide 56}
This is another TEM. On the right there is a NP image with that Titan microscope.

\subsubsection{Slide 57}
What is the actual most powerful microscope on the earth? It is the LHC.

\subsubsection{Slide 58}
We will use SEM. The instrument is more compact and so it is simpler in principle to be used.

\subsubsection{Slide 59}
This is the instrument that we will use in our lab activity.
Inside there is ultra high vacuum.
You have different detectors to collect the signal produced.

It has a field emission source.

\subsubsection{Slide 60}
The main advantage of SEM versus TEM is that he can operate with bulk sample. That is you do not need to specially prepare the sample, but you can take a bulk sample and put in the chamber provided that it can sustain high vacuum conditions. In principle there is no need of a priori preparation of the sample.

In TEM you have to achieve electron trasparency to obtain trasmittion trough the sample, because electrons interact strongly with matter. So to achieve electrons trasparency you need to stay below 100 nm at most, so your local thickness should be as low as possible for obtain a good quality TEM image.


You can obtain information on morphology, topography and composition. The beam energy is normally below \( 30keV \).

There are special kind of SEM, environmental SEM, which can work with low vacuum conditions.

The resolution is normally around 1nm. It improves if you work with eleveated energy and with FEG sources.

\subsubsection{Slide 61}
Block diagram of a SEM.
An electron gun produces electrons which are focused by condenser lenses. Then there are the scan coils which are able to raster your beam into a prescibed pattern and to deflect the beam with a controlled deflection. Then you have the objective lens which produces the finest tip  at your sample level.

At this point you will collect a lot of signals. You have the \textbf{back scattering electron (BSE)} detectors (elastically scattering electrons), the phtonmultipler of the \textbf{secondary electrons} (which collects electrons produced by inelastic interaction in your sample). Then we can also obtain informations with X-rays signals for instance produced by the interaction of electrons with the matter.

On the right there is the typical SEM sketch.

One of the major limitations is the charging effect. For instance, if your sample is insulating, the arrival of electrons produce charges build up. So this charging effect is not good for good quality images. In the past you need to cover insulating materials with a tiny layer of carbon or gold, but at the expenses of the quality of the image.
Nowadays, since we are able to obtain good quality image with quite low electron current (or energies) we can dramatically reduced the charged build up without affecting too much the quality of the image. So that we can image even non-conductive materials without pre-processing of the sample.



\subsubsection{Slide 62}
The most important point of electron microscop is the source.

On the past the most commonc sources were the thermoionic gun, in which filament is heated to a sufficiently high temperature so that the thermoionic effect starts to occur. The Richardson law tells us that the current density is proportional to the temperature squared times the exponential which contains the temperature. \( E_w \) is the work function of the filament material, tipically tungesten or LaB6 (typical thermoionic source).

Nowadays, modern micrscope use field emission gun in which instead of a simple heating of the filament (which occurs at very high temperature,from several hundres of degree), you can now work with a high polarization so that electrons can be extracted by tunneling effect and not just by thermoionic effect.


\subsubsection{Slide 63}

So you can obtain a system like that, in which if the work function is depicted in the sketch on the left (where there is 0 as the energy of vacum and we have in blue the condution band of the metal). The work function is the energy that you need to give to the system to bring an electrons in the vacuum level.
In this case it is \( W=-E_F \), so minus the Fermi energy.


If you polarize your filament with respect to the environment, you can band the vacuum level (sketch on the right). \( x \) is the distance from the surface of your material, the electric field is constant so that the different of potential \( V \) goes linear with the coordinate and so you can band this vacuum level, so that electrons can be emitted with a much higher efficiencies with a tunneling effect, directly from a fermi level of your material.

So this process requires slower temperatures, but very intense electrins field (10V/m).
You need to have much much better vacuum in your system, otherwise unwanted charges will damage all the system.
So much better coherent properties and brillance of your system brightness can be obtained.

\subsubsection{Slide 64}
One you have produced electrons, you need to deflect them. This is the major point which prevents X-rays to be used as candidate for microscopy.

The typical way to deflect electrons in modern electronmicroscope is to use a solenoid. It is basically a system in which you can run a current, with this coils so you can produce a very intense magnetic field inside the solenoid itself (first figure from the top).

If you coat the solenoid with a magnetic material (second figure from the top) you can guide the magnetic fields lines exactly in a very tiny region in which you want to have your electric and magnetic field confined.

If you shape the polar pieces in a way that you can even betetr guide the magnetic field lines in a lower region (third figure from the top), you can very intense magnetic field with a very good control of the spatial distribution and so you can use the Lorentz force to deflect electrons very easily.

You could use condenser as well capacitors, but the effectivness of electric deflection is not so good as the one for magnetic deflection. So the magnetic lenses are nowadays the most commonly used.

\subsubsection{Slide 65}
There is a typical magnetic lens with iron coating on the left. You can produce very easily a bell shape (cylindrical symmetric magnetic field) in a very near region in the direction in which electrons are coming.

The typical variation of the magnetic field along the z direction which is the axial direction of electrons is typically \( B(z) \).

If you consider the Lortentz force you obtain the formula in this slide. You have a circolar motion composed with a linear motion, the result is an elicoidal motion.

You can calculate the focus length of this lens. Keep in mind that you can easily deflect electrons and focus electrons with this technique.

\subsubsection{Slide 66}

Once you have produced and deflected electrons, at this point you need to rast them (send them exactly in the place where you want to interact with the sample).

The typical approach of the scanning electron microscope is to scan your with the scan coil along lines to cover the entire surface that you want to probe. So that you can divide into pixel your images and collect the signal from each individual pixel, so that you can reconstruct the signal produced when the beam was a that specific pixel and reconstruct the image or the property which is spatial dependent for your specific material.

You can produce the images sequentially.

The resolution is ultimately related to the beam size: the smallest is the beam, the smallest is the surface that you can probe per pixel. So the magnification is the ration between the length of the area scanned and the length of the samples.

If you want to scan your samples with these pixels, if your beam is lower than the pixel that is a meaningfull image (because each pixel produces a different image). Whereas if you use a pixel which is smaller than the beam size, you will convolve. Even if apparently you are using smaller pixels than the beam size, your information is lost within the beam size. So that this is a sort of magnificatiom without resolution which is not useful. So the minimum pixel should be equal to the beam size, otherwise you have loss of information.


\subsubsection{Slide 67}
This is the sketch of gemini column in our microscope.
Descrpition of that sketch in this slide.

\subsubsection{Slide 68}
This is a sketch of the magnification. If the sample and the image have the same size, the magnification is 1. If you reduce the rastered area, we can enlarge the magnification, because now the same number of pixel is obtained with a smaller area in the samples.

\subsubsection{Slide 69}
Let us see the signal produced in the SEM. When you send an electron beam toward your material, electrons enter in your material and will be scattered. They will produce very complicated trajectories. Overall, they will interact in a volume with respect to the sample, which is called the \textbf{interaction volume}.

According to the depth in the material, we will produce different signals.

The secondary electrons are the most important for electron microscopy (produced by inelastic scattering).

The backscattered electrons are produced by elastic scattering.

By analyzing X-rays we can obtain informations on the composition of the sample.

Globally the interaction volume is 1 \( \mu m  \) in size.
One could think that this value should be the resolution of our system, but that is not true. If we have a correlated signal with the position, we can assign to that position a variation in the signal. So that ultimately we obtain information with a resolution which is controlled by the beam size.

The typical resolution for SEM is of the order of 1nm.

\subsubsection{Slide 70}
Just to have a first feeling the difference in the production of secondary electrons with back scattered electrons.

If we send a primary electron in our beam, interacting with an atom in our material, it will produce an ionizion of the inner shell. This inner electron will be ejected in a form of secondary electron.
This will create a hole in that shell.

On the contrary, if the primary electron is backscattered or deflected toward the same direction of the incoming beam, we speak about elastic scattering and we have more energetic electrons. The two process have different probability. With high atomic number the backscattered is more favoured, but these two processes occurs in all materials with different efficiencies.

\subsubsection{Slide 71}
If we look at the depth distribution of the signal produced, since secondary electrons are produced at lower energies, they can escape from shallower region with respect to the surface. So when secondary electrons impinges on a structured object, the largest number of secondary electrons are produced at grazing incidence, with respect to normal incidence. So that the region (1) will produce more electrons, so it will be brigther with respect to (2).

On the contrary, with backscattered electrons the contrast is produced by the difference in mass. If you have an inomhogeneous sample of silicon and gold (like in our case), the atomic number is very different, so the backscattering cross-section is the Rutherford cross-section which is proportional to the atomic number and so that the gold layer will be able to produce much much more intense signal in backscattering than silicon.

So backscattered normally is used to produce compositional constrast, whereas secondary electrons is used to obtain morphological or structural or topografical contrast.

Contrast is the most important concept in imaging. If I want to see something, I need to produce contrast between different region. If our material will be perfectly homogeneous with no structures on the surface, it will be structureless for us, because we cannot see any difference in the material. On the contrary, we can produce contrast if we have topographical contrast or atomic mass contrast.

\subsubsection{Slide 72}
Those are experimental images of the very same sample. On the left we have the image with secondary electrons, in which we have these very detailed topografical contrast.

Since the material is homogeneous, but fractured on the surface we will have a very poor imaging with backscattering, because there is no major information because the material is homogeneous.

\subsubsection{Slide 73}
To better the topogrphical contrast in fractured surface, you see the effect of this image is more or less as a 3D effect, or if we were using the standard visible line coming from the direction indicated.

Normally, when you see this effect is because you are using a detector which is set off axis with respect to he incoming beam, as in the first figure from the top on the right.
The contrast will be dominated by the large number of electrons produced toward the detector.

So when you see brigher region, this means that the detector is spatially in this direction, for understanding the mechanism of image formation in this particular case.


\subsubsection{Slide 74}
In this case we have a superlattice of 3-5 seminconductors, with different thickness.

If you produce a AlAs as a lower average atomic mass with respect GaAs, in backscattered the brigher layer are made of GaAs, which is the thinner.

So you can clearly see that you have a very good resolution with a high compositional contrast.

In the figure on the left, you have the equivalent secondary electrons image of the  very same structure, in which the contrast appears reversed. In this case is a topographical contrast, because the sample was prepared by thinnig the polishing in the sample producing a contrast in the typographic, when passing from a material to another. You see in this case that you cannot clearly see the nature of the material.

So by combingin secondary electrons with back scattered electrons, you can obtain a very complete information in terms of composition, morphology and topography in your system.

\subsubsection{Slide 75}
Other examples of other nanostructures materials. Those are the precursor of of gold nano-shells: polystere sphere partially covered with gold nanoparticles.

On the left you have the secondary electrons, on the right the backscattered image. In this way (BES) you have the contrast dominated by the particles, while in SE the contrast is more dominated by the topography of the nanostructures.

So according to the imaging mode you can obtain different combinations of informations.


\subsubsection{Slide 76}
Those are the nanoparticles produced by your collegues in the previous lab activities. There are gold nanoparticles in a silycon substrate. This is image is made with secondary electrons. In this case there is no major andvantage in working with backscattered electrons, because the contrast is already very good, because we have a reflect surface and all the contrast comes from the nanoparticles.

\subsubsection{Slide 77}
If we zoom a little bit in, we can directly see the nanoparticles. Here you see aggregates of particles, but it is not due to the aggregation within the solution, but it is due to the producedure that we used to attach them to the silycon  surface in which we used the APTES functionalization, in a way to glue them to the silycon surface.

What you will do is to perform a similar analysis on the samples, that you will receive in the electron microspy session, so that you can perform the analysis of the size distribution.

\subsubsection{Slide 78}
This is a comparison between secondary electrons and backscattered electrons in a very large aggregate of nanopartciles in our system. Of course in the SE the contrast is better, while in the BSE everything is blury because the material is homogeneous and so you cannot directly spot the individual nanoparticles.
This is a clear indication that you have carefully select your image mode to enhance the visibility of the feature that you need to investigate.



\end{document}
