\documentclass[../main/main.tex]{subfiles}

\newdate{date}{08}{04}{2020}

\begin{document}

\section{Lecture 9}
 \displaydate{date}. Compiled:  \today. Alice.

\subsubsection{Slide 123}

\begin{figure}[h!]
\centering
\includegraphics[page=1,width=0.9\textwidth]{../lessons/pdf_file/9_lesson.pdf}
\end{figure}

Features of ion implantation. It is useful to explore the concept of nucleation and growth. So it is not used only to produce a supersaturated solid solution to obtain precipitation of nanoparticles but also to process and to obtain a clever control over the nanostructures to modify them in a clever way to obtain novel functionalities.

\newpage \subsubsection{Slide 124}

\begin{figure}[h!]
\centering
\includegraphics[page=2,width=0.9\textwidth]{../lessons/pdf_file/9_lesson.pdf}
\end{figure}

Ion implantation can be described in a simple picture: you accelerate an ion trough a potential difference so that the ions will reach kinetic energies of the order of 10-100keV up to 100MeV or GeV (normally the typical range is keV-Mev because you want to convince those ions to enters any kind of substrate to modify the properties of the material).

Since we are working ion by ion it is a typical technique based on bottom-up approach. For that reasion we use the ion implanted sample as a prototipical example of those techniques for produce supersaturated solid solution (in this case) to explore the concepts that we have seen so far.

\newpage

\subsubsection{Slide 125}

\begin{figure}[h!]
\centering
\includegraphics[page=3,width=0.9\textwidth]{../lessons/pdf_file/9_lesson.pdf}
\end{figure}

The main features of ion implantation are brifly mentioned here in this list (but there are other interesting concepts).

In principle, since the technique works with no thermodynamic equilibrium because you are working  outside the solubility limit. You can introduce any material in any substrate in principle without suffering the limitation of the solubility, so you can implant up to this higher concentration and so you can hevily modify the substrate according to the required properties. So you can easily produce supersaturation (you are not constrained by solubility limit)

You can obtain a nice control over the depth at which we are depositing our implanted species (you can tune the energy of the implanted species to reach that depth you need).

Another interesting property is the patterning, a capability of ion implantation. Since you are using a sort of beam which can be focused, you can have a movement of your beam so you can raster over an area to obtain an uniform implantation area. Also you can implant trough a mask which produce a contrast in which you deposit selectively the ions which you want to implant in specific region. This is very important and ion implantation is the most used technique for instance in electronics for producing C-MOS and transistors.

Ion implantation is widely used even if is quite expensive technique because it requires larges infrastructure as we will see. It is typical used for instance for semiconductive doping...but also to produce nanostructures or to modify them.

In general you can control quite nicely the doping of your material since you are basically working with very low current density, so there is no high increasing temperature. When you want to diffuse an element the temperature should go up and this could have a strong effect on the junction for instance (you can obtain a diffusion of bad species). SO ion implantation has not this problem and it is very effective for producing junctions.

Moreover, we can obtain trough multiple ion implantation the formation of compunds and since we are working outisde the thermodynamic equilibrium you can produce nanostructures which are not obtainable in the bulk phase. For instance if we have two species which can not be mixed in the bulk phase (gold and cobalt, because the phase diagram at room temperature tells you that you should have separated phases), this is no longer true at the nanoscale for one side because you have the surface which alters the usual thermodynamic equilibrium but since you are not working under thermodynamic equilibrium you are not forced to obey those contraint and so you can easily produce a metastable alloys which cannote be obtained in the bulk.


The last important feature is the capability of ion-beam processing: capability to modify a material through the controlled release of energy through the implanted species.

\newpage

\subsubsection{Slide 126}

\begin{figure}[h!]
\centering
\includegraphics[page=4,width=0.9\textwidth]{../lessons/pdf_file/9_lesson.pdf}
\end{figure}

\begin{enumerate}
\item In this sketch, we would like to show hot to deal with ion-beam processing. For instance suppose to have a transparent glass substrate (ultimately we are interested in optical properties and all the example in this course are based on trasparent silica substrate). The easyest way to do ion implantation is to form nanoparticles according to the production of supersaturated solid solution through ion implantation.

\item You can also obtain a controlled release of ions entering the substrate. Ions will release energy entering the substrate with to basic mechanicsm: electronic energy loss or nuclei energy loss. If you work with light ions you are in the lectronic regime. If you have for instance a solid line glass substrate which contains a lot of ions, which can be ion exchanged through suitable technique called \textbf{ion-exchange process}. So if you put your solid line glass containing for instance sodium or other metals in a bath containing silver ions for instance, at sufficiently high temperatures \( 300° \) you are able to create a concetration gradient in which the external silver atoms enter your substrate whereares the internal sodium ions go out.
There is a ion-exchange process in which you can uniformly dop your glass with the silver ions and after that to promote the precipitation instead of using a thermal process, you can use the energy released by an additional beam (for instance Helium, Neon or other light ions) which goes through the exchanged layer in which you have basically the disperse ions. The energy release is sufficient to unify those ions and to form nanoparticles.

\item The last class of ion-bem processing is this. For instance if you want to obtain clustering or a mixing of two phases, instead of producing directly an alloying or a mixed system you can produce a multilayer structure (element1, element2, element2...in a stack)  but you want to obtain a mixed structure and nanostructured system. You can use heavy ion radiation (ion with high mass) which are able to breake the structures and to release energy to mix-up the ordered structures and to promote mixing of the two interfaces up to a certain level in which you have precipitation and formation even of alloyed system.

\end{enumerate}

\newpage

\subsubsection{Slide 127}

\begin{figure}[h!]
\centering
\includegraphics[page=5,width=0.9\textwidth]{../lessons/pdf_file/9_lesson.pdf}
\end{figure}

The typical ions beams is produced by ion implanter in Legnaro.
The low fluence is used for doping semiconductors, the very high fluence is used to obtain large precipitation of nanostructures. In the case of Indium clusters in silica we may remember that the fluence is closed to the maxium fluence obtainable by our system.

The simple sketch of an ion implanter is this. You have an ion source (ions are produced by a plasma treatment of solid source, or from gas source). Then we have pre acceleration system in which you can reach the energy that you need for your system.
Then we have a magnect which acts as a sort of prison, which control the ratio charge of mass of ions by changin the magnetic field. So basically you can select with high accuracy the mass of your implanted species and direct them to the implantation channel in which we have a beam deflectors in \( x \) and \( y \) direction (area) and then you can reach your implantation chamber in which you have your substrate that you want to implant.

All the system is under vacuum, because you do not want to have unwanted species or scattering of your ions.
You can control also the temperature of your implanted samples because you can control the mobility, diffusivity etc...

\newpage

\subsubsection{Slide 129}

\begin{figure}[h!]
\centering
\includegraphics[page=6,width=0.9\textwidth]{../lessons/pdf_file/9_lesson.pdf}
\end{figure}

If we consider the kind of implantable species that currently we are using in our lab, they are highlithed in the picture. Our core business on the metallic species since we want to produce system for optics, nanophotonics.

The elements \( C,N,Si \) are useful for the processing of the surfaces.

It is important the capability to implant lantanids as \( Er \) (ermium) is of fundamental revelance for nanophotonics and in particular for fibreottic technology or telecomunications, in which you need to use those ions to pump and to inject light in the wavelength of telecomunication (\( 1.5 \mu \)).

\newpage

\subsubsection{Slide 130}

\begin{figure}[h!]
\centering
\includegraphics[page=7,width=0.9\textwidth]{../lessons/pdf_file/9_lesson.pdf}
\end{figure}

Let us remind the most important features of binary collision that occur when you are sending an ion toward the substrate. The substrate will acquire a momuntum and deflected, the same for ions. The process obey conservation of momentum and energy.

This is the first collision, but a cascade of collision can be produced. You will produce a lot of defects and incoming ions will suffer random collision which produce a random walk of incoming ions toward the surface.

The intaraction between the two atoms can be modelled using a coulomb interaction where the \( \chi  \) is a \textbf{screening function}, which represent te level of screening of the electronic shell that modulated the interaction.
This is the Thomas-Fermi version for \( \chi  \) but it is not important.

One relevant quantity is the transferred kinetic energy in the process (this is the formula in the center of mass reference system for describing the collision), where \( \theta  \) is the scattering angle in the center of mass of the particle 1 with respect to the direction of motion.

This angle is related to the angle in the lab reference frame by this simple relation.

You can maximize \( T_E \) by using the same mass for the two species. In that basis you can have a sort of head-on collision in which the first atom basically will sit on the position of the other atom which is set in motion whereas the other is at rest in its original position (?).

Typically the minimum energy required to obtain such a defect in the substrate is that the transferred kinetic energy is larger than the typical binding energy (usually is of few eV) of the substrate.
So for an energy of order of KeV you have a lot of possibility to transfer high amount of energy to atoms to produce a collisional cascade.

Eventually, the atom are removed from the original lattice position or they even being removed from the material (they can be sputtered at the surface and removed).

\newpage

\subsubsection{Slide 131}

\begin{figure}[h!]
\centering
\includegraphics[page=8,width=0.9\textwidth]{../lessons/pdf_file/9_lesson.pdf}
\end{figure}

Just to have a quantitatively description let us see this example.
In this example we describe boro implantation in silica that is the typical doping process for semiconductors industry. The collision will produce scattering at \( \theta = 45° \) in the lab reference.

The silicon atoms will receive a lot of energy to be used for subsequent interactions with other atoms in the substrate. In that case, since the first atom will have the very same mass of the other atoms in the substrate (same composition, silicon with silicon) the energy transfer will be very efficient. There will be a lot of damage produced by the first interaction.

\newpage

\subsubsection{Slide 132}

\begin{figure}[h!]
\centering
\includegraphics[page=9,width=0.9\textwidth]{../lessons/pdf_file/9_lesson.pdf}
\end{figure}

The energy distribution within the system as follow up from this collision cascade can be controlled by simple concept which is the \textbf{energy loss} in the system.

When the incoming ions enters the system will start to lose energy  in a typical rate of 0.1-1 keV/m. This energy can be divided in two terms: the first one due to the coulombian interaction with the screening nucleian charge (\textbf{elastic energy loss}) and the other which is the \textbf{inelastic contribution} which stands for the interaction with the electronic cloud (electronic exitation of the system). The physical quantity normally used for describing this energy loss process is the stopping cross section which is the energy loss function normalized by the number of diffusing centers in the target.

We can recast this equation in terms of the energy loss function and the stopping cross section which is divided in the elastic and inelastic component.

What you can do is to follow the evolution of these two quantities in the system whereas the incoming ions travel within the substrate.

\newpage

\subsubsection{Slide 136}

\begin{figure}[h!]
\centering
\includegraphics[page=10,width=0.9\textwidth]{../lessons/pdf_file/9_lesson.pdf}
\end{figure}

This is a very simple sketch of the process (you can implant at another angle instead of 90° to modulate the penetration depth in your system).

You have an ion implanted normally to the substrate. The ion will be suffered by binary collision so it will scatter all around when travelling inside the substrate and it will finally stop at this point in which its energy is completely faded and so it has to stop.
The projection of this path with respect its original direction of motion is called the \textbf{projected range} (the path is the range). We are interested in the projected value instead of the range, because if we send a large number of ions they will reach different depth but we are interested in the average value of those quantities.

A simple way to calculate the range is to integrate from the initial energy of the incoming ion up to the energy zero (atom stop) the inverse of the energy loss function in \( \dd[]{E}  \) (related to the concept of stopping cross section). The range will be the length of the path described by the ions within the substrate.

The projected range is the average of all the projected ranges of the entire set of atoms sent toward the substrate. This can be written as a function of the typical ion range \( R \) divided by these coefficients incoming ion mass \( M_1 \) and substrate atomic mass \( M_2 \) (in the simple assumption that you have a monoatomic substrate, but it can be very easily generalized for describing multi component substrate just considering all the possible subset of couple ion-atom in the system).

The other important quantity is the dispersion with respect to the projected range \( R_p \) which is called \textbf{straggling} which is the half width half maximum of the distribution of the projected ranges of each individual implanted ions.

\newpage

\subsubsection{Slide 137}

\begin{figure}[h!]
\centering
\includegraphics[page=11,width=0.9\textwidth]{../lessons/pdf_file/9_lesson.pdf}
\end{figure}

Let us look at the typical implantation profile (this is not quite rigorous approach but quite nicely followed by the typical implantation condition), we can assume that the distribution profile of the concentration profile with respect to depth can be described by a Gaussian-like profile in which the centroid of the Gaussian function is the average projected range and the sigma value is the straggling.

We can define the \textbf{fluence} which is the integral of the concentration with respect to the depth.

We can do simulations for calculating those quantities (stocastic description of the interaction and with montecarlo you sen a lot of ions to the substrate and you collect very different impact parameters and you can basically follow up what is the exact atomistic position of each individual atom in the system as a function of the implantation energy etc...).

\newpage

\subsubsection{Slide 138}

\begin{figure}[h!]
\centering
\includegraphics[page=12,width=0.9\textwidth]{../lessons/pdf_file/9_lesson.pdf}
\end{figure}

This is a simple example for a light element implantation into silica.

If we calculate the energy loss function we can divide it into the two components. The electronic curve is indistiguishable from the total energy loss because the major contribution to the total electronic energy loss in this case comes from the electronic part, while the nuclean component is very very small.

The kinematical properties of this implantation are described in the second plot. There are two curves in which you have the projected range and the straggling.

If we are able to grow up in energy we can reach even several microns in the implanted substrate. However we usually work at \( 100 \) keV hence we have a reduced range.

\newpage

\subsubsection{Slide 139}

\begin{figure}[h!]
\centering
\includegraphics[page=13,width=0.9\textwidth]{../lessons/pdf_file/9_lesson.pdf}
\end{figure}

Not let us consider another example with gold (which has a larger mass as before) in silica.

In this case the largest contribution is the nuclear one. The electronic contribution start to dominate at larger implantation energy (swift heavy ions regime). It is important that you can control the balance between elastic and inelastic contribution of energy loss.

In this case the projected range is much slower because the mass is larger.

\newpage

\subsubsection{Slide 140}

\begin{figure}[h!]
\centering
\includegraphics[page=14,width=0.9\textwidth]{../lessons/pdf_file/9_lesson.pdf}
\end{figure}

This is the gold implantation. The actual cross section image where there is superimposed the expected concentration profile which was simulated by a program (called \textbf{SRIM}) which has been used to calculate the quantities in the previous slide.

You clearly see that the concentration reached here is quite hight with respect to the absolute concentration.

In this case we are not exactly in the case of homogeneous distribution of concentration (as underlined) but we have a large gradient of concentration. We can avoid this unhomogeneous doping by using a very simple scheme which is a multiple ion implantation of the very same species but with different energies. Since the energy is directly proportional to the projected range, increasing the energy we can obtain another gaussian profile in the depth.  If we have for istance two other implantation (so we have a triple implantation scheme) we can obtain a sort of box like profile in which we can obtain a better control of the uniformity of concentration in our system. We used this scheme for obtaining the controlled deposition of gold nanoparticles and we were able to control the size of nanoparticles at the atomic level. So we are able to produce a controlled precipitation of atoms for forming ultra small gold nanoparticles with the number of atoms ranging from 10 atoms up to 20-25 atoms per classed (?).

This will imply additional complication in the nanofabrication but in principle it can be handle quite easily.

\newpage

\subsubsection{Slide 141}

\begin{figure}[h!]
\centering
\includegraphics[page=15,width=0.9\textwidth]{../lessons/pdf_file/9_lesson.pdf}
\end{figure}

As mentoned when you increase the energy you basically reach different projected range, but corresponding the straggling is larger so the distribution is more spreade. The lowest is the energy the sharpest is the peak. So you can use even this technique for producing layer doping of you material (produce very shallow implanted region very close to the surface).

Those are example computed by SRIM for different implantation energies at the same fluence (dose).

\newpage

\subsubsection{Slide 142}

\begin{figure}[h!]
\centering
\includegraphics[page=16,width=0.9\textwidth]{../lessons/pdf_file/9_lesson.pdf}
\end{figure}

This is a compare of implantation of golden and copper at the very same energy and fluence, but since they have different masses the two projected range are significantly different and also the straggling.

If you want to produce an alloy between gold and copper you want to have a spatial overlap of the two distribution, so you should for instance reduce the implantation energy of copper until  the projected range matches the one of gold implantation.
In that case you have the largest overlap between the two distribution concentration, so you have the largest chance to obtain alloyed gold copper nanoparticle. This is a very clever and easy way to obtain alloyed nanostructured systems (?).

\newpage

\subsubsection{Slide 143}

\begin{figure}[h!]
\centering
\includegraphics[page=17,width=0.9\textwidth]{../lessons/pdf_file/9_lesson.pdf}
\end{figure}

Those are basic definition for defect configurations.

\newpage

\subsubsection{Slide 144}

\begin{figure}[h!]
\centering
\includegraphics[page=18,width=0.9\textwidth]{../lessons/pdf_file/9_lesson.pdf}
\end{figure}

These are some definition of energies.

\begin{itemize}
\item \textbf{Displacement energy}. Energy required to displace a target atom far enough away from its original lattice size, so he cannot re-enter in ints original site.
The result of this displacement is the creation of \textbf{Frekel Pair}.

\item \textbf{Lattice Binding energy}. The minimum energy to remove an atom from the lattice site.
This is a lower energy with respect to the displacement energy.

\item \textbf{Surface Binding energy}. Binding energy of atoms at the surface, so it is the energy required to remove a surface atom and it will be lower. So the surface binding energy is typycally lower than the lattice binding energy.

This produce an effect reported in the plot here. We have the concentration in logarithmic scale with respect to the depth. In this case the concentration is no longer a gaussian profile because when you remove atoms from the target you obtain that the first atoms are deposeted at that projectedrange, but the very next ions (when you have sputtered layer) will have the very same average projected range. But since the surface is sputtered to some extent that projected range will be shifted so that you have that the net result will be an accumulation on the surface of your atom and so you have no longer a gaussian like profile but a sort of more flat profile. This will be the case when you have a large surface sputtering.\footnote{In physics, sputtering is a phenomenon in which microscopic particles of a solid material are ejected from its surface, after the material is itself bombarded by energetic particles of a plasma or gas.} (?)

\end{itemize}

\newpage 

\subsubsection{Slide 145}

\begin{figure}[h!]
\centering
\includegraphics[page=19,width=0.9\textwidth]{../lessons/pdf_file/9_lesson.pdf}
\end{figure}

In this simulation we would like to describe the net effect of the damage produced by an incoming ions.
Molecular dynamics is used to follow the evolution of an implantation which is gold ions into an ordered gold coppered alloyed.

This is the cross section of the system and you have to imagine that your ion is implanted perpendicular to this image.
We see that immediately few pico seconds after the arrival of the ions the net effect is the creation of disorder in the system (circular section). You see that the atom then try to recover the original position (yellow atoms are gold, while red are copper). The net effect of the interaction of a single ion of gold is the production of this diffective configuration in the system.

The system will try to self anneal the damage. The minimum potential energy of the interaction between different atoms will try to reset the crystalline order, but you see that the process is not sufficient. Indeed, the structural ordering is quite nice but the chemical ordering is not preserved (in this case we have a particular effective interaction).

You have this wave of expansion, because in the initial stages of implantation you have a lot of energy released by the incoming ions which produces a local net increasing of temperature. The effective temperature can be as large as hundred or thousand of equivalent kelvin degree. This is just a local temperature.

The atoms around the implanted trajectory, which is exactly in the center, are basically uneffected and so they remain are there normal temperature. So the global temperature does not increase, but the local temperature can be very very large.

The ions produce an \textbf{ion track} that is a track within the sample in which damage is produced. You have to imagine a sort of cilinder following the range of the path of your incoming ions in the system. Cylindric volume in which energy is released to the largest extent.

\newpage

\subsubsection{Slide 146}


\begin{figure}[h!]
\centering
\includegraphics[page=20,width=0.9\textwidth]{../lessons/pdf_file/9_lesson.pdf}
\end{figure}

Now, let us try to understand what is going on. The one of before was a plane view this is a corss section view of the implantation of gold in silica. The red trajectory is the trajectory followed statistically by gold atoms, and green and blue atoms are silica and oxigen atoms displaced by the incoming ion.

You see that even if gold stops here, the dimension extents more of less two times than the projected range.
You can obtain damge up to three times the projected range in particular situations.

This is the effect of a single atom simulated by the SRIM program.

\newpage

\subsubsection{Slide 147}

\begin{figure}[h!]
\centering
\includegraphics[page=21,width=0.9\textwidth]{../lessons/pdf_file/9_lesson.pdf}
\end{figure}

If we had 10 additional ions, we can follow the trajectory of the implanted species (ions) and in blue and green the trajectories of the displaced atoms in the substrate.

As seen in the molecular dinamic calculation, in the range of pico seconds with respect to the initial beam, we have a recover of the matrix that is the self annealing process due to the energy of the atoms tends to reconstruct the initial structure.

The process is not perfect, so if you want to achieve a full recovery you need to perform a thermal annealing in our system and so you need to go to additional strategies for recovering damages.

\newpage

\subsubsection{Slide 151}

\begin{figure}[h!]
\centering
\includegraphics[page=22,width=0.9\textwidth]{../lessons/pdf_file/9_lesson.pdf}
\end{figure}

A clever use of the energy released by atoms is reported in these graphs. We would like to describe an experiment did for ion beam processing.

The experiment was aimed at obtained the formation of alloyed golden copper system. For modulating the optical properties of the nanoparticles, we end up in this system seen in cross section (fig 1). By the way, this is the sample described briefly when we mentioned the correlated diffusion process.


In this case (fig1) we have those beautifull clusters made of golden copper in a given composition. If we perform the thermal annealing we obtain the correlated motion of copper and gold you see a very very far from the implanted range.
This is another description of the Onsager correlated diffusion. The composition of the yellow cluster and red clusters are the same just the correlated motion of the two species triggered by the external motion of the incoming species from the atmosphere of annealing. (?)

What is important is that if we anneal the sample we can produce the clusters with different sizes according to the local concetration (we have the largest clusters at the projected range and accumulated surface at the surface due to the diffusion outward and inward). But what is interesting is the fact that we can produce those beutiful tiny clusters (in blue) at the surface of original nanoclusters by performing a Neon ion implantation. Neon ions are implanted so that they will stop well beyond the cluster distribution. They produce sort halo of tiny particles around each original nanoparticle.
The fig.3 is a zoom of the previous one and we clearly see that the original cluster 1 is sorrounded by an halo of tiny nanoparticles which are very interesting for optical properties.

In this case we have used ion beam as a proccessing tool for promoting nanofabrication an production of interesting structure.

We see that the shape of the particle is no longer spherical as the original structure but is elongated along the direction of the implantation of the Neon ions.
We have a distribution of tiny clusters,which are alloyed clusters (made out of gold and copper), even if with the slightly different composition with respect to the original one.
The basic phenomenon here is a sort of sputtering of atoms from a nanostructure source (the incoming ions are able to extract atoms from the solid nanoparticles, which receive a sufficient energy to decorate the original cluster).

Indeed, the distribution is not symmetric with respect to the original surface, because in the red part we have a lesser number of clusters (on the original incoming surface) and a larger number in the opposite direction, where the momentum transfer is maximum.

To understand what is going on in this process it takes a lot of experiments and simulations.
Ultimately what we ended up in a clear picture what is going on when an incoming ion is interacting with an already formed nanostructure.

\newpage

\subsubsection{Slide 152}

\begin{figure}[h!]
\centering
\includegraphics[page=23,width=0.9\textwidth]{../lessons/pdf_file/9_lesson.pdf}
\end{figure}

We investigated in large details the process to achive the controlled production of those nanocluster around the central core. To do this controlled synthesis we investigated how we can exploit the two regimes of energies release that we have during ion implantation. They are the nuclear stopping power (or the nuclear energy loss) and the electronic energy loss.

To do that we changed the mass of the implanted ions, so we changed the energy in order to have a large electronic contribution. In the case of Helium implantation to reach the other regime in which nuclear elastic component dominates we implanted kripton double ionized.

We have varied the fluence, the current density in order to obtain a range which was almost identical in the four implantation experiment and which is 3 times the projected range of the species implanted to form nanoparticles.
In this pecular experiment we have used golden and silver ato obtain gold and silver nanoallyed nanoparticles.

We varied the fluence to keep constant the enrgy density release to the system. This is important to better compare the situation for a given energy density release and a given power density release (energy release in the very same amount of time). SO there is no heating effect which can promote differences between the four experiments.

We end up in the plot, where \( Z \) is the atomic number of the implanted species and we have the ratio between elastic and nuclear stopping power.

\newpage

\subsubsection{Slide 153}

\begin{figure}[h!]
\centering
\includegraphics[page=24,width=0.9\textwidth]{../lessons/pdf_file/9_lesson.pdf}
\end{figure}

The result of those experiment are reported in these plots.

\begin{itemize}
\item The first one is the sample implanted with the Helium. In this case we have the alloyed nanoparticle and we see that the original particles are sorrounded by an halo of tiny nanoparticle. There is an asymmetric distribution of halos which is larger in the lower part with respect to the surface. This is a balistic process in which we preferentially sputter the atoms from the clusters in the direction of motion of the incoming beam.

\item By looking at the experiment with neon ions at higher energy and lower fluence and lower current density (so that we preserve the energy and power released during the experiment). We see that the density of nanoparticle is larger in this case.

\item If we go with higher energy in implantation with higher mass that is kripton double ionized, we see that the original cluster is dramatically affected by this implantation and even the smaller nanoparticles exibit a quite larger density of satellite halos around the central core.

The satellite starts to be of the same size of the original core and the satellite are crystalline (as you can see from the high resolution trasmission electron microscopi in which you see the atomic structure of the satellites).

\end{itemize}

Eventually this process can be controlled since we are able to carefully control the energy release to our system and with ion implantation this is quite easy to do in a controlled way.

\newpage

\subsubsection{Slide 154}

\begin{figure}[h!]
\centering
\includegraphics[page=25,width=0.9\textwidth]{../lessons/pdf_file/9_lesson.pdf}
\end{figure}

Ultimately we can obtain a controlled transformation from a spherical nanoparticle like in this case, by ion irradiation to these very interesting geometry of core+satellite nanoparticles.

The net effect is that we can control the number of atoms extracted and therefore the number of precipitated satellite nanoscluster around the original core.

In fig.2 and fig.3 we have changed the fluence (the radiation time) for a given implantation species and a given implantation energy, we can control the kinetic of formation of those nanoparticles.

The main message is the fact that we can obtain with this kinetic process something which is the reverse of the Ostwald Rippening. That is instead of promoting the growth of the largest clusters, we decompose the largest clusters to smaller nanoparticles. This process is also called \textbf{inverse Ostwald rippening} and is triggered by the non thermodynamic equilibrium under which ion radiation operates.

We can as well describe this process as the selective extration of atoms from the original cluster as a sort of nanosource for ion implantation. The source is the sputtered atom from the original cluster which is spread radially around the central core in a very controlled way.

We basically can control the distance from the original nanoparticle, the size, the density.

\newpage

\subsubsection{Slide 155}

\begin{figure}[h!]
\centering
\includegraphics[page=26,width=0.9\textwidth]{../lessons/pdf_file/9_lesson.pdf}
\end{figure}

In this experiment we described the evolution of the satellite diamater as a function of the fluence which is basically the time coordinate at which we are looking at the experiment.

The maximum distance from the central core is a function which goes as \( t^{1/2} \), which is a typical diffusive process.

The radius of the core classes decreases as \( t^{1/3} \) (typical of the Ostwald rippening but in the inverse case).
To preserve mass concentration the radius of the satellite cluster increase with the same exponential \( t^{1/3} \).

This is a remarkable example in how we can obtain a control thermodynamic at the nanoscale with ion radiation.

The important message is that since we are able to control those parameters we are able to control the position of the clusters, their size and eventually when we will look at optical properties of these nanostructures we will see the importance to control those parameters to control the global properties of those systems.

\newpage

\subsubsection{Slide 156}

\begin{figure}[h!]
\centering
\includegraphics[page=27,width=0.9\textwidth]{../lessons/pdf_file/9_lesson.pdf}
\end{figure}

Such class of experiments can be generalized to build like in this case a matrix of deformation of spherical nanoparticles triggered by the ion implantation. We got the following.
Starting from spherical particles with increasing size and by increasing the fluence of the radiation (or changing the energy, the ions ans so on...that is controlling the amount of energy release and proportion between nuclear and electronic component) in this beautiful paper people were able to show that we can transform for instances spherical particles into road.

This is because as mentioned in the experiment with the animation described by molecular dynamics, when an incoming beam pass trough the nanoparticles basically the local temperature can be soo high that the nanoparticles and the matrix sorround can be melted! Along the ion track (path followed by the incoming ion) we have a preferential melting of the structure which promotes diffusion in the liquid phase (which is faster than diffusion in the solid phase) so that the melted atoms instead of reforming a sphere, they formed an elongated structure because the track is more or less of this size. When the local solid resolidification occur, you end up in this elongated nanoparticles.

The degree of elongation is controlled under mass conservation and what is really interesting is that you can obtain such remarkable transformation which can be very useful to change the asymmetry in the particle. This can be large affect the optical properties.

You can obtain a large zoology of results.








\clearpage





\end{document}
